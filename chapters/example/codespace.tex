\section{Codespace}
\subsection{Minted}
\emph{Important: Minted requires python Pygments and the \mintinline{text}{--shell-escape} flag.}

\begin{itemize}
  \item External import

  \inputcode[highlightlines={1,10-13}]{Python}{code/example.py}

  \item With custom line range

  \inputcode[firstnumber=1,firstline=10,lastline=13]{Python}{code/example.py}

  \item Embedded

  \begin{code}{python}
  from typing import Iterator

  # This is an example
  class Math:
      @staticmethod
      def fib(n: int) -> Iterator[int]:
          """Fibonacci series up to n."""
          a, b = 0, 1
          while a < n:
              yield a
              a, b = b, a + b

  result = sum(Math.fib(42))
  print("The answer is {}".format(result))
  \end{code}

  \item Inline

  \mintinline{Python}{print('Hello, world!')}
\end{itemize}

\subsection{Algorithms}
This is one way to input algorithms.

\begin{algorithm}[H]
  \caption{QL algorithm}
  Initialize \(Q\)-table values \((Q(s, a))\) arbitrarily\;
  Initialize a state \((s_t)\)\;
  Repeat Steps~\ref{alg:step_4} to~\ref{alg:step_6} until learning period ends\;
  Choose an action \((a_t)\) for the current state \((s_t)\) using an exploratory policy\; \nllabel{alg:step_4}
  Take action \((a_t)\) and observe the new state \((s_t + 1)\) and reward \((r_t + 1)\)\;
  Update \(Q\)-value\; \nllabel{alg:step_6}
\end{algorithm}
