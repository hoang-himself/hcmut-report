\section{Better tables}
The recommended way is by using the booktabs package and drop all vertical rules.

Tabularx is simply tabular but with X environment, meaning that it will try to use all of \mintinline{latex}{\linewidth}.

\begin{center}
  \begin{tabularx}{\linewidth}{l*{2}{X}}
    \toprule
         & OOP & FP \\
    \cmidrule(lr){2-3}
    Pros &     &    \\
         &     &    \\
         &     &    \\
    \midrule
    Cons &     &    \\
         &     &    \\
         &     &    \\
    \bottomrule
  \end{tabularx}
\end{center}

More information can be found at \url{https://latex-tutorial.com/tables-in-latex/}.

In \mintinline{latex}{tabular}, with the \mintinline{latex}{p}, \mintinline{latex}{m} or \mintinline{latex}{b} column types, sometimes you will notice that the width of the table is wider than the sum of the widths of the columns.
This is due to the padding added by the \mintinline{latex}{\tabcolsep} and the line width of the vertical separators which are added by default.

\begin{code}{latex}
  \tabcolsep + p{length} + \tabcolsep
\end{code}

By default, \mintinline{latex}{tabcolsep} is set to 6pt, which equals to 2.12mm in digital printing.
The use of \mintinline{latex}{@{}..@{}} voids this behavior.
