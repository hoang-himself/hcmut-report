\documentclass{article}

%%%%%%%%%%%% Global layout %%%%%%%%%%%%
\usepackage%
[
    a4paper,
    % left=2cm,
    % right=2cm,
    % top=2cm,
    % bottom=2cm,
    % vmargin=2cm % vertical margins
    % hmargin=3cm % horizontal margins
    margin=3cm
]
{geometry}
\usepackage{fancyhdr, graphicx, hyperref, indentfirst, lastpage, setspace}

%%%%%%%%%%%% Encodings %%%%%%%%%%%%
\usepackage[utf8]{vntex, inputenc}
\usepackage{amsmath, amssymb, gensymb, textcomp}

%%%%%%%%%%%% Bibliography %%%%%%%%%%%%
\usepackage{fvextra, csquotes} % This causes T5 encoding warnings; can be ignored
\usepackage%
[
    backend=biber,      % Sort
    style=alphabetic,   % Check URL in biblio.tex
    sorting=ynt         % Check URL in bilbio.tex
]
{biblatex}
\usepackage[english]{babel}

%%%%%%%%%%%% Better table %%%%%%%%%%%%
\usepackage{array, booktabs, multicol, multirow, siunitx, tabularx}
% Wide tables go here https://tex.stackexchange.com/questions/332902/my-table-doesnt-fit-what-are-my-options

%%%%%%%%%%%% Better enumerate %%%%%%%%%%%%
\usepackage{enumitem}

%%%%%%%%%%%% Code space %%%%%%%%%%%%
\usepackage[dvipsnames]{xcolor}
\usepackage{tikz}
\usepackage[framemethod=tikz]{mdframed}
\usepackage{minted, verbatim} % needs python Pygments and --shell-escape flag

%%%%%%%%%%%% Graphics %%%%%%%%%%%%
\usepackage{caption, float}

% Allow setting >max< width of figure
\usepackage[export]{adjustbox} % 'export' allows adjustbox keys in \includegraphics
\usepackage{mwe} % TODO Remove sample images

%%%%%%%%%%%% Subpreambles %%%%%%%%%%%%
% Possible, won't do
% https://tex.stackexchange.com/a/395416

%%%%%%%%%%%% Global style setup %%%%%%%%%%%%
% Bibliography
\addbibresource{./ref.bib}
\nocite{*} % Print listed bibliography even if they are not cited; * means all

% Hyperlinks
\hypersetup%
{
  colorlinks=true,
  urlcolor=blue,
  linkcolor=black,
  citecolor=red
}

% Others
\renewcommand{\arraystretch}{1.2} % vertical space between table rows
\usemintedstyle{emacs} % pygment/code color format

\allowdisplaybreaks{} % to have page breaks inside align* environment

\numberwithin{equation}{section} % related to numbering format
% {equation}{section} means <section number>.<equation number>
% {equation}{subsection} means <section number>.<subsection number>.<equation number>

% \everymath{\color{blue}} % Makes a lot of things blue, only use if absolutely necessary

\AtBeginDocument{\renewcommand*\contentsname{Contents}} % Table of contents section text
\AtBeginDocument{\renewcommand*\refname{References}} % References section text

%%%%%%%%%%%% Header and footer style %%%%%%%%%%%%
\pagestyle{fancy}
\setlength{\headheight}{40pt}

\makeatletter
\renewcommand\Huge{\@setfontsize\Huge{22pt}{18}} % change font size for cover
\makeatother

\renewcommand{\headrulewidth}{0.3pt}
\renewcommand{\footrulewidth}{0.3pt}

% Header
\fancyhead{} % clear all header fields
\fancyhead[L]{
  \begin{tabular}{rl}
    \begin{picture}(25,15)(0,0)
    \put(0,-8){\includegraphics[width=8mm, height=8mm]{./assets/hcmut.png}}
    \end{picture}
    \begin{tabular}{l}
      \textbf{\bf \ttfamily University of Technology, Ho Chi Minh City}\\
      \textbf{\bf \ttfamily Faculty of Computer Science and Engineering}
    \end{tabular}
  \end{tabular}
}
\fancyhead[R]{
	\begin{tabular}{l}
		\tiny \bf \\
		\tiny \bf
	\end{tabular}
}

% Footer
\fancyfoot{} % clear all footer fields
\fancyfoot[L]{\scriptsize \ttfamily Report for XXX --- Academic year 20xx--20xx} % TODO subject and year
\fancyfoot[R]{\scriptsize \ttfamily Page {\thepage}/\pageref{LastPage}}

%%%%%%%%%%%% Custom commands and environments %%%%%%%%%%%%
\newcommand*\mean[1]{\bar{#1}}

\newenvironment{code}[1]
{\VerbatimEnvironment%
  \begin{mdframed}[leftline=false,rightline=false,backgroundcolor=magenta!10,nobreak=false]%
    \begin{minted}[linenos=true,breaklines,breaksymbolleft=,obeytabs=true,tabsize=2]{#1}%
}
{
    \end{minted}%
  \end{mdframed}%
} % TODO Sample codeblock

%%%%%%%%%%%% Main %%%%%%%%%%%%
\begin{document}

\begin{titlepage}
    \begin{center}
        VIETNAM NATIONAL UNIVERSITY, HO CHI MINH CITY \\
        UNIVERSITY OF TECHNOLOGY \\
        FACULTY OF COMPUTER SCIENCE AND ENGINEERING
    \end{center}

    \vspace{1cm}

    \begin{figure}[H]
        \centering
        \includegraphics[width=0.5\textwidth]{./assets/hcmut.png}
    \end{figure}

    \vspace{1cm}

    \begin{center}
        \begin{tabular}{c}
            \textbf{\Large COURSE (COURSE\_CODE)}                     \\ % TODO Course information
            {}                                                        \\
            \midrule                                                  \\
            \textbf{\Large Report (Semester 2xx, Duration: xx weeks)} \\ % TODO Report times
            {}                                                        \\
            \textbf{\Huge Title}                                      \\ % TODO Report title
            {}                                                        \\
            \bottomrule
        \end{tabular}
    \end{center}

    \vspace{3cm}

    \begin{table}[h]
        \begin{tabular}{rl}
            \hspace{1cm} Advisor:      & Mr.\ Boombastic \\ % TODO Instructor information
                                       &                 \\
            \hspace{1cm} Student Name: & Nguyễn Hoàng    \\ % TODO Student name
            \hspace{1cm} Student ID\@: & 1952255         \\ % TODO Student ID
        \end{tabular}
    \end{table}

    \begin{center}
        {\footnotesize HO CHI MINH CITY, DECEMBER 2021} \\ % TODO Time
    \end{center}
\end{titlepage}

% \thispagestyle{empty} % skip page number

\section*{Member list \& Workload}
\begin{center}
    \begin{tabular}{llcc} % TODO Group information
        \toprule
        \textbf{No.} & \textbf{Full name} & \textbf{Student ID} & \textbf{Percentage of work} \\
        \midrule
        1            & Nguyễn Hoàng       & 1952255             & 100\%                       \\
        \bottomrule
    \end{tabular}
\end{center}

\newpage
\tableofcontents
\newpage

\section{Pre-ambles in sub-files}
Using preambles in included files are possible, but I will not implement it, at least for now.
Read more at \url{https://tex.stackexchange.com/a/395416}.

\section{Normal section}
This is how you normally work with \LaTeX, but you can also split a project into smaller files for easier management.
To import other files, you can use \mintinline{latex}{\input{}} or \mintinline{latex}{\include{}}.
There differences can be found at \url{https://tex.stackexchange.com/a/250}, but in short

\[\mintinline{latex}{\include{filename}} = \mintinline{latex}{\clearpage \input{filename} \clearpage}\]

\section{Better tables}
The recommended way is by using the booktabs package and drop all vertical rules.

Tabularx is simply tabular but with X environment, meaning that it will try to use all of \mintinline{latex}{\linewidth}.

\begin{center}
  \begin{tabularx}{\textwidth}{l*{2}{X}}
    \toprule
         & OOP & FP \\
    \cmidrule(lr){2-3}
    Pros &     &    \\
         &     &    \\
         &     &    \\
    \midrule
    Cons &     &    \\
         &     &    \\
         &     &    \\
    \bottomrule
  \end{tabularx}
\end{center}

More information can be found at \url{https://latex-tutorial.com/tables-in-latex/}.


\section{Better enumerator}
Normal enumerator gets the job done, but what if you want custom numbering?
This implementation allows custom labeling, either by pre-defined rules or in-place.

\begin{enumerate}[start=4,label={\alph*.yeah}]
  \item First item
  \item Second item
  \item[custom] Third item
\end{enumerate}


\section{Codeblocks}
There are several ways to embed code in a \LaTeX file.
Here I demonstrate inline code, embedded codeblocks, and external import.

This version of embedded code uses a custom environment defined in the main file.
You can expand this environment instead of using \mintinline{latex}{code} as well.

\begin{code}{Python}
  class iostream:
      def __lshift__(self, other):
          print(other, end='')
          return self

      def __repr__(self):
          return ''


  if __name__ == "__main__":
      cout = iostream()
      endl = '\n'
      cout << "Hello" << ", " << "World!" << endl
\end{code}

\begin{mdframed}[leftline=false,rightline=false,backgroundcolor=magenta!10,nobreak=false]
  \inputminted[linenos=true,breaklines,breaksymbolleft=,obeytabs=true,tabsize=2]{Python}{main.py}
\end{mdframed}

You can also define your custom inline as \url{https://tex.stackexchange.com/a/148479}.


\section{Figures with flexible width}

\hrule % to see \linewidth

\includegraphics[max width=0.9\linewidth]{example-image-1x1}

\includegraphics[max width=0.9\linewidth]{example-image-a4}

With \mintinline{latex}{\adjincludegraphics} (or \mintinline{latex}{\adjustimage}) you can also use the original width as \mintinline{latex}{\width}:

\adjincludegraphics[width=\ifdim \width > \linewidth \linewidth \else \width \fi]{example-image}


\section{Bibliography}

Using \texttt{biblatex} you can display bibliography divided into sections, depending of citation type.
Let's cite! The Einstein's journal paper \cite{einstein} and the Dirac's book \cite{dirac} are physics related items.
Next, \textit{The \LaTeX\ Companion} book, the Donald Knuth's website \cite{knuthwebsite}, \textit{The Comprehensive Tex Archive Network} (CTAN) are \LaTeX\ related items; but the others Donald Knuth's items \cite{knuth-fa} are dedicated to programming.

Learn more at \url{https://www.overleaf.com/learn/latex/Bibliography_management_in_LaTeX}


\printbibliography

\end{document}
