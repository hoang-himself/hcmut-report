\documentclass[twoside]{hcmut-report}
\usepackage{codespace}

% Sub-preambles
% https://github.com/MartinScharrer/standalone

% Encodings
\usepackage{gensymb,textcomp}

% Better tables
% Wide tables go to https://tex.stackexchange.com/q/332902
\usepackage{array,multicol,multirow,siunitx,tabularx}

% Better enum
\usepackage{enumitem}

% Graphics
\usepackage{caption,float}

% Configurations
\coursename{Course name h}
\reporttype{Report type h}
\title{Report title h}
\advisor{Advisor h}

% Rename some sections
%\AtBeginDocument{\renewcommand*{\contentsname}{Contents}}
%\AtBeginDocument{\renewcommand*{\refname}{References}}
%\AtBeginDocument{\renewcommand*{\bibname}{References}}

% Custom commands
\newcommand*\mean[1]{\bar{#1}}

\begin{document}
\coverpage%

%\section*{Member list \& Workload}
%\newcounter{memberrowno}
%\setcounter{memberrowno}{0}
%\begin{center}
%  \begin{tabular}{>{\stepcounter{memberrowno}\thememberrowno}llcc}
%    \toprule
%    \multicolumn{1}{c}{\textbf{No.}} & \textbf{Full name} & \textbf{Student ID} & \textbf{Contribution} \\
%    \midrule
%                                     & h                  & xxxxxxx             & 100\%                       \\
%                                     & h                  & xxxxxxx             & 100\%                       \\
%    \bottomrule
%  \end{tabular}
%\end{center}
%\clearpage

\tableofcontents

\clearpage
\section{Normal section}
This is how you normally work with \LaTeX, but you can also split a project into smaller files for easier management.
To import other files, you can use \mintinline{latex}{\input{}} or \mintinline{latex}{\include{}}.
There differences can be found at \url{https://tex.stackexchange.com/a/250}, but in short

\begin{center}
  \mintinline{latex}{\include{filename}} = \mintinline{latex}{\clearpage \input{filename} \clearpage}
\end{center}

\section{Better tables}
The recommended way is by using the booktabs package and drop all vertical rules.

Tabularx is simply tabular but with X environment, meaning that it will try to use all of \mintinline{latex}{\linewidth}.

\begin{center}
  \begin{tabularx}{\linewidth}{l*{2}{X}}
    \toprule
         & OOP & FP \\
    \cmidrule(lr){2-3}
    Pros &     &    \\
         &     &    \\
         &     &    \\
    \midrule
    Cons &     &    \\
         &     &    \\
         &     &    \\
    \bottomrule
  \end{tabularx}
\end{center}

More information can be found at \url{https://latex-tutorial.com/tables-in-latex/}.

In \mintinline{latex}{tabular}, with the \mintinline{latex}{p}, \mintinline{latex}{m} or \mintinline{latex}{b} column types, sometimes you will notice that the width of the table is wider than the sum of the widths of the columns.
This is due to the padding added by the \mintinline{latex}{\tabcolsep} and the line width of the vertical separators which are added by default.

\begin{code}{latex}
  \tabcolsep + p{length} + \tabcolsep
\end{code}

By default, \mintinline{latex}{tabcolsep} is set to 6pt, which equals to 2.12mm in digital printing.
The use of \mintinline{latex}{@{}..@{}} voids this behavior.

\section{Better enumerator}
Normal enumerator gets the job done, but what if you want custom numbering?
This implementation allows custom labeling, either by pre-defined rules or in-place.

\begin{enumerate}[start=4,label={\alph*.yeah}]
  \item First item
  \item Second item
  \item[custom] Third item
\end{enumerate}

\section{Codespace}
\subsection{Minted}
\emph{Important: Minted requires python Pygments and the \mintinline{text}{--shell-escape} flag.}

\begin{itemize}
  \item External import

  \inputcode[highlightlines={1,10-13}]{Python}{code/example.py}

  \item With custom line range

  \inputcode[firstnumber=1,firstline=10,lastline=13]{Python}{code/example.py}

  \item Embedded

  \begin{code}{python}
  from typing import Iterator

  # This is an example
  class Math:
      @staticmethod
      def fib(n: int) -> Iterator[int]:
          """Fibonacci series up to n."""
          a, b = 0, 1
          while a < n:
              yield a
              a, b = b, a + b

  result = sum(Math.fib(42))
  print("The answer is {}".format(result))
  \end{code}

  \item Inline

  \mintinline{Python}{print('Hello, world!')}
\end{itemize}

\subsection{Algorithms}
This is one way to input algorithms.

\begin{algorithm}[H]
  \caption{QL algorithm}
  Initialize \(Q\)-table values \((Q(s, a))\) arbitrarily\;
  Initialize a state \((s_t)\)\;
  Repeat Steps~\ref{alg:step_4} to~\ref{alg:step_6} until learning period ends\;
  Choose an action \((a_t)\) for the current state \((s_t)\) using an exploratory policy\; \nllabel{alg:step_4}
  Take action \((a_t)\) and observe the new state \((s_t + 1)\) and reward \((r_t + 1)\)\;
  Update \(Q\)-value\; \nllabel{alg:step_6}
\end{algorithm}


\bibliographystyle{plain}
\bibliography{refs/example.bib}
\nocite{*}

\end{document}
