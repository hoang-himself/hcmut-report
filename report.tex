\documentclass{article}

\usepackage{hcmut-report,codespace}

% Encodings
\usepackage{amsmath,amssymb,gensymb,textcomp}

% Better tables
% Wide tables go https://tex.stackexchange.com/q/332902
\usepackage{array,multicol,multirow,siunitx,tabularx}

% Better enum
\usepackage{enumitem}

% Graphics
\usepackage{caption,float}

% Allow setting >max< width of figure
% Remove if unneeded
\usepackage[export]{adjustbox} % 'export' allows adjustbox keys in \includegraphics
\usepackage{mwe}

% Custom commands
\newcommand*\mean[1]{\bar{#1}}

\ocoursename{Course name h}
\oreporttype{Report type h}
\title{Report title h}
\oadvisor{Advisor h}
\reportlayout%

\begin{document}

\coverpage%

\section*{Member list \& Workload}
\begin{center}
  \begin{tabular}{llcc}
    \toprule
    \textbf{No.} & \textbf{Full name} & \textbf{Student ID} & \textbf{Percentage of work} \\
    \midrule
    1            & h                  & xxxxxxx             & 100\%                       \\
    \bottomrule
  \end{tabular}
\end{center}

\newpage
\tableofcontents
\newpage

\section{Normal section}
This is how you normally work with \LaTeX, but you can also split a project into smaller files for easier management.
To import other files, you can use \mintinline{latex}{\input{}} or \mintinline{latex}{\include{}}.
There differences can be found at \url{https://tex.stackexchange.com/a/250}, but in short

\[\mintinline{latex}{\include{filename}} = \mintinline{latex}{\clearpage \input{filename} \clearpage}\]

\section{Better tables}
The recommended way is by using the booktabs package and drop all vertical rules.

Tabularx is simply tabular but with X environment, meaning that it will try to use all of \mintinline{latex}{\linewidth}.

\begin{center}
  \begin{tabularx}{\textwidth}{l*{2}{X}}
    \toprule
         & OOP & FP \\
    \cmidrule(lr){2-3}
    Pros &     &    \\
         &     &    \\
         &     &    \\
    \midrule
    Cons &     &    \\
         &     &    \\
         &     &    \\
    \bottomrule
  \end{tabularx}
\end{center}

More information can be found at \url{https://latex-tutorial.com/tables-in-latex/}.


\section{Better enumerator}
Normal enumerator gets the job done, but what if you want custom numbering?
This implementation allows custom labeling, either by pre-defined rules or in-place.

\begin{enumerate}[label={\alph*.yeah}]
  \item First item
  \item Second item
  \item[custom] Third item
\end{enumerate}


\section{Codeblocks}
There are several ways to embed code in a \LaTeX{} file.
Here I demonstrate inline code, embedded codeblocks, and external import.

This version of embedded code uses a custom environment defined in the main file.
You can expand this environment instead of using \mintinline{latex}{code} as well.

\begin{code}{Python}
  class iostream:
      def __lshift__(self, other):
          print(other, end='')
          return self

      def __repr__(self):
          return ''


  if __name__ == "__main__":
      cout = iostream()
      endl = '\n'
      cout << "Hello" << ", " << "World!" << endl
\end{code}

\begin{mdframed}[leftline=false,rightline=false,backgroundcolor=magenta!10,nobreak=false]
  \inputminted[linenos=true,breaklines,breaksymbolleft=,obeytabs=true,tabsize=2]{Python}{assets/main.py}
\end{mdframed}

You can also define your custom inline as \url{https://tex.stackexchange.com/a/148479}.

This is one way to input algorithms.

\begin{algorithm}
  \caption{QL algorithm}
  Initialize \(Q\)-table values \((Q(s, a))\) arbitrarily\;
  Initialize a state \((s_t)\)\;
  Repeat Steps~\ref{alg:step_4} to~\ref{alg:step_6} until learning period ends\;
  Choose an action \((a_t)\) for the current state \((s_t)\) using an exploratory policy\; \nllabel{alg:step_4}
  Take action \((a_t)\) and observe the new state \((s_t + 1)\) and reward \((r_t + 1)\)\;
  Update \(Q\)-value\; \nllabel{alg:step_6}
\end{algorithm}


\section{Figures with flexible width}

\hrule % to see \linewidth

\includegraphics[max width=0.9\linewidth]{example-image-1x1}

\includegraphics[max width=0.9\linewidth]{example-image-a4}

With \mintinline{latex}{\adjincludegraphics} (or \mintinline{latex}{\adjustimage}) you can also use the original width as \mintinline{latex}{\width}:

\adjincludegraphics[width=\ifdim \width > \linewidth \linewidth \else \width \fi]{example-image}


\bibliographystyle{plain}
\bibliography{refs}
\nocite{*}

\end{document}
